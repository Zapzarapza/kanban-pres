\documentclass[12pt]{beamer}

\usepackage{darkdefault}
%\usetheme{Singapore}

% no pagenumbers
\pagenumbering{gobble}

\usepackage[utf8]{inputenc}
\usepackage[english]{babel}
\usepackage{graphicx}
\usepackage{hyperref}
\usepackage{amssymb} % for checkmarks
\usepackage{array}
\usepackage{makecell}

\usepackage{caption}
\renewcommand{\figurename}{FIGURE}
\captionsetup[figure]{labelfont={color=hilight}}

\hypersetup{colorlinks,linkcolor=hilight, urlcolor=lolight}

\newcommand{\src}[2]{\href{#1}{\color{lolight}#2}}


% hide text when using \pause
\setbeamercovered{invisible}

% title info
\title{Software Kanban}
\subtitle{A Visual Process-Managment System for Software Development}
\author{{\footnotesize Marko Oreskovic, Kevin Rabe, Andreas Ohmer,\\ Sebastian Müller \& Alexander Tkachov}}
\institute{Frankfurt University of Applied Sciences}
\date{\small 24th November 2016}


% a few macros
\newcommand{\bi}{\begin{itemize}}
\newcommand{\ei}{\end{itemize}}

\newcommand{\be}{\begin{enumerate}}
\newcommand{\ee}{\end{enumerate}}

\newcommand{\ig}{\includegraphics}
\newcommand{\subt}[1]{{\scriptsize \color{subtitle} {#1}}}

\newcommand{\tick}{\checkmark}
\newcommand{\dash}{--}

%\renewcommand{\arraystretch}{1.5}

\begin{document}
	
	\maketitle
	
	\begin{frame}{Table of Contents}
		\tableofcontents[hideallsubsections]
	\end{frame}
	
	
		
		
%	\section{General Overview}
%	 
%		\begin{frame}{What is Kanban?}{General Overview}
%			\bi
%				\item process-management system in software
%				\item invented in Japan in the 50s
%				\item visualised with a Kanban-Board
%			\ei
%			
%			{\color{orange}Vielleicht mehr Zeug einfügen?}
%		\end{frame}
		
		
	\section{History of Kanban}
	
%		\begin{frame}{\secname}
%			\begin{figure}
%				\ig[scale=0.55]{pictures/YesWeKanban.jpg}
%				\caption{Taiichi Ohno, father of Kanban}
%			\end{figure}
%			\subt{Source: \src{https://www.limitedwip.org}{Limited WIP Society}}
%		\end{frame}

		\begin{frame}{\secname}{General Information}
			\bi
				\item Kanban = signboard
				\item Developed by Taiichi Ohno in 1947
				\item Worked for Toyota
				\item Scheduling system for lean manufacturing and just-in-time manufacturing
			\ei
		\end{frame}
		
		\begin{frame}{\secname}{Taiichi Ohno}
			\begin{figure}
				\ig[scale=0.8]{pictures/taiichi}
				\caption{Taiichi Ohno, father of Kanban (1912 - 1990)}
			\end{figure}
		\end{frame}
		
		\begin{frame}{\secname}{Goals and Reasons}
			Reasons:
			\bi
				\item Too high storage cost
				\item Too little productivity
				\item Increasing customer requirements
			\ei
			
			\vspace{0.5cm}
			\pause
			
			Goal:
			\bi
				\item Steady flow in the production process \\ $\Rightarrow$ less inventory needed
			\ei
		\end{frame}
		
		\begin{frame}{\secname}{Kanban in SWE}
			\bi
				\item First use in 2004 by Microsoft
				\item Lean Software Development
				\item First public presentation in 2007 by David J. Anderson
				\item Anderson: Father of Software Kanban
			\ei
		\end{frame}
		
	\section{Kanban outside of Software Development}
	
	\section{Kanban in Software Development}
	
		\begin{frame}{\secname}
			\begin{figure}
				\ig[scale=0.4]{pictures/agile_lean.png}
				\caption{Kanban has elements of Agile and Lean SWE}
			\end{figure}
		\end{frame}
		
		\begin{frame}{\secname}{Agile Software Engineering}
			Values:
			\bi
				\item Individuals and interactions over processes and tools
				\item Working software over comprehensive documentation
				\item Customer collaboration over contract negotiation
				\item Responding to change over following a plan
			\ei
		\end{frame}
		
		\begin{frame}{\secname}{Agile Software Engineering}
			\begin{figure}					
%				\begin{tabular}{m{2.3cm}  m{0.6cm} m{1.5cm} m{2.1cm} m{2.3cm}}
				\renewcommand{\arraystretch}{2}
				\begin{tabular}{c|cccc}
					& Pull & \makecell{Limited \\Tasks} & \makecell{Transp. \\ Information} & \makecell{Cont. \\ Improvement} \\ \hline
					\makecell{Individuals \& \\ Interactions} & \tick & \dash & \tick & \tick \\ \hline
					\makecell{Working \\Software} & \dash & \dash & \dash & \tick \\ \hline
					\makecell{Customer \\ Collaboration} & \dash & \dash & \dash & \dash \\ \hline
					\makecell{Responding \\to Change} & \tick & \tick & \tick & \tick \\					
				\end{tabular}
				\caption{Values of Agile SWE and Kanban}
			\end{figure}
		\end{frame}
		
		\begin{frame}{\secname}{Lean Software Engineering}
			Values:
			\bi
				\item Estimate Waste
				\item Amplify Learning
				\item Decide as late as possible
				\item Deliver as fast as possible
				\item Empower the Team
				\item Build Integrity in
				\item See the whole
			\ei
		\end{frame}
		
		\begin{frame}{\secname}{Lean Software Engineering}
			\begin{figure}					
				%				\begin{tabular}{m{2.3cm}  m{0.6cm} m{1.5cm} m{2.1cm} m{2.3cm}}
				\renewcommand{\arraystretch}{1.3}
				\begin{tabular}{c|cccc}
					& Pull & \makecell{Limited \\Tasks} & \makecell{Transp. \\ Information} & \makecell{Cont. \\ Improv.} \\ \hline	
					\makecell{Eliminate Waste}  & \tick & \tick & \tick & \tick \\ \hline	
					\makecell{Amplify Learning} & \tick & \tick & \tick & \tick \\ \hline
					\makecell{Decide as Late \\ as Possible} & \tick & \tick & \tick & \dash \\ \hline
					\makecell{Deliver as Fast \\ as Possible} & \tick & \tick & \tick & \dash \\ \hline
					\makecell{Empower the Team} & \tick & \dash & \tick & \tick \\ \hline
					\makecell{Build Integrity in} & \dash & \dash & \tick & \tick \\ \hline
					\makecell{See the Whole} & \dash & \dash & \dash & \tick
				\end{tabular}
				\caption{Values of Lean SWE and Kanban}
			\end{figure}
		\end{frame}
		
		
	\section{The Kanban-Board}
	
		\begin{frame}{\secname}{Variations}
			\begin{figure}
				\ig[scale=0.11]{pictures/board_simple}
				\caption{A very basic Kanban-Board}
			\end{figure}
%			\subt{Source: \src{https://upload.wikimedia.org/wikipedia/commons/d/d3/Simple-kanban-board-.jpg}{Wikipedia}}
		\end{frame}
	
		\begin{frame}{\secname}{Variations}
			\begin{figure}
				\ig[scale=0.25]{pictures/var1.png}
				\caption{Kanban-Board with more sections}
			\end{figure}
%			\subt{Source: \url{https://www.youtube.com/watch?v=ndWPFk7GR8k}}
		\end{frame}
		
		\begin{frame}{\secname}{Variations}
			\begin{figure}
				\ig[scale=0.25]{pictures/var2}
				\caption{Same Kanban-Board in Action}
			\end{figure}
%			\subt{Source: \url{https://www.youtube.com/watch?v=ndWPFk7GR8k}}
		\end{frame}
		
		\begin{frame}{\secname}{Variations}
			\begin{figure}
				\ig[scale=0.28]{pictures/var3}
				\caption{Kanban-Board with limits}
			\end{figure}
%			\subt{Source: \url{https://www.youtube.com/watch?v=ndWPFk7GR8k}}
		\end{frame}
		
		\begin{frame}{\secname}{Other useful Information}
			\begin{figure}
				\ig[scale=0.29]{pictures/Durchsatz}
				\caption{Throughput}
			\end{figure}
%			\subt{Source: \url{https://www.youtube.com/watch?v=ndWPFk7GR8k}}
		\end{frame}
		
		\begin{frame}{\secname}{Other useful Information}
			\begin{figure}
				\ig[scale=0.26]{pictures/Durchlaufzeit}
				\caption{Cycle Time}
			\end{figure}
%			\subt{Source: \url{https://www.youtube.com/watch?v=ndWPFk7GR8k}}
		\end{frame}
		
		\begin{frame}{\secname}{Other useful Information}
			\begin{figure}
				\ig[scale=0.3]{pictures/Fehlerrate}
				\caption{Rate of Errors}
			\end{figure}
%			\subt{Source: \url{https://www.youtube.com/watch?v=ndWPFk7GR8k}}
		\end{frame}
		
	\section{Benefits of Kanban}
	
	
	
		\begin{frame}{Conclusion}
			Kanban is \\
			
			\bi
				\item easy to learn
				\item versatile
			\ei
			
			{\color{orange}Mehr Zeug einfügen!!!}
		\end{frame}
		
		
		\begin{frame}{Sources}
			\bi
				\item Epping, Thomas: \textit{Kanban für die Softwareentwicklung.} Springer-Verlag 2011
				\item \url{https://www.youtube.com/watch?v=ndWPFk7GR8k}
				\item \href{https://upload.wikimedia.org/wikipedia/commons/d/d3/Simple-kanban-board-.jpg}{Wikimedia}
			\ei
		\end{frame}
	
		
	
\end{document}
