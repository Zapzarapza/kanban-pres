\documentclass[12pt]{beamer}

\usepackage{darkdefault}
%\usetheme{Singapore}

% no pagenumbers
\pagenumbering{gobble}

\usepackage[utf8]{inputenc}
\usepackage[english]{babel}
\usepackage{graphicx}
\usepackage{hyperref}

\usepackage{caption}
\renewcommand{\figurename}{FIGURE}
\captionsetup[figure]{labelfont={color=hilight}}

\hypersetup{colorlinks,linkcolor=hilight, urlcolor=lolight}

\newcommand{\src}[2]{\href{#1}{\color{lolight}#2}}


% hide text when using \pause
\setbeamercovered{invisible}

% title info
\title{Software Kanban}
\subtitle{A Visual Process-Managment System for Software Development}
\author{{\footnotesize Marko Oreskovic, Kevin Rabe, Andreas Ohmer,\\ Sebastian Müller \& Alexander Tkachov}}
\institute{Frankfurt University of Applied Sciences}
\date{\small 24th November 2016}


% a few macros
\newcommand{\bi}{\begin{itemize}}
\newcommand{\ei}{\end{itemize}}

\newcommand{\be}{\begin{enumerate}}
\newcommand{\ee}{\end{enumerate}}

\newcommand{\ig}{\includegraphics}
\newcommand{\subt}[1]{{\scriptsize \color{subtitle} {#1}}}


\begin{document}
	
	\maketitle
	
	\begin{frame}{Table of Contents}
		\tableofcontents[hideallsubsections]
	\end{frame}
	
	
		
		
	\section{General Overview}
	 
		\begin{frame}{What is Kanban?}{General Overview}
			\bi
				\item process-management system in software
				\item invented in Japan in the 50s
				\item visualised with a Kanban-Board
			\ei
			
			{\color{orange}Vielleicht mehr Zeug einfügen?}
		\end{frame}
		
		
	\section{History of Kanban}
	
		\begin{frame}{\secname}
			\begin{figure}
				\ig[scale=0.55]{pictures/YesWeKanban.jpg}
				\caption{Taiichi Ohno, father of Kanban}
			\end{figure}
			\subt{Source: \src{https://www.limitedwip.org}{Limited WIP Society}}
		\end{frame}
		
	\section{Kanban outside of Software Development}
	
	\section{Kanban in Software Development}
	
		\begin{frame}{\secname}
			\begin{figure}
				\ig[scale=0.4]{pictures/agile_lean.png}
				\caption{Kanban has elements of agile and lean SWE}
			\end{figure}
		\end{frame}
		
	\section{The Kanban-Board}
	
		\begin{frame}{\secname}{Variations}
			\begin{figure}
				\ig[scale=0.1]{pictures/board_simple}
				\caption{A very basic Kanban-Board}
			\end{figure}
			\subt{Source: \src{https://upload.wikimedia.org/wikipedia/commons/d/d3/Simple-kanban-board-.jpg}{Wikipedia}}
		\end{frame}
	
		\begin{frame}{\secname}{Variations}
			\begin{figure}
				\ig[scale=0.21]{pictures/var1.png}
				\caption{Kanban-Board with more sections}
			\end{figure}
			\subt{Source: \url{https://www.youtube.com/watch?v=ndWPFk7GR8k}}
		\end{frame}
		
		\begin{frame}{\secname}{Variations}
			\begin{figure}
				\ig[scale=0.2]{pictures/var2}
				\caption{Same Kanban-Board in Action}
			\end{figure}
			\subt{Source: \url{https://www.youtube.com/watch?v=ndWPFk7GR8k}}
		\end{frame}
		
		\begin{frame}{\secname}{Variations}
			\begin{figure}
				\ig[scale=0.21]{pictures/var3}
				\caption{Kanban-Board with limits}
			\end{figure}
			\subt{Source: \url{https://www.youtube.com/watch?v=ndWPFk7GR8k}}
		\end{frame}
		
	\section{Benefits of Kanban}
	
	
	
		\begin{frame}{Conclusion}
			Kanban is \\
			
			\bi
				\item easy to learn
				\item versatile
			\ei
			
			{\color{orange}Mehr Zeug einfügen!!!}
		\end{frame}
	
		
	
\end{document}
